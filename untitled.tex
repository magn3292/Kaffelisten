\documentclass{article}

% in the preamble go all packages and commands  *NOT* compatible with Authorea but necessary to offline compiling

%\CopyrightYear{2016}

\input{header}

\title{Kaffelisten}

\input{authors}

\begin{document}

\thispagestyle{empty}

\vspace{0.1cm}
\begin{center}
%\includegraphics[width=1.0\textwidth]{bg}

\newcommand{\HRule}{\rule{\linewidth}{0.5mm}}

\textsc{\LARGE University of Southern Denmark}\\[0.7 cm] % Name of your university/college
\textsc{\Large Experts in Teams}\\[0.4cm] % Major heading such as course name
\textsc{\large 5th Semester, Autumn 2016}\\[0.4cm] 
\textsc{\large Theme: \emph{Formula Student} (Theme 4) }\\[0.4cm] 
% Minor heading such as course title
\textsc{\large Team Name: \emph{Dynamic ! Business} (Team 6)}\\[0.4cm] 
\textsc{Revision: 1.1 }\\[0.4cm] 
\HRule \\[0.3cm]
{ \huge \bfseries Team Contract}\\[0.2cm] % Title of your document
\HRule \\[1.0cm]

\begin{tabular}{ll}
\makebox[2.8in]{\hrulefill} & \makebox[1.8in]{\hrulefill}\\
William Bergmann Børresen & Date\\[6ex]
\makebox[2.8in]{\hrulefill} & \makebox[1.8in]{\hrulefill}\\
Jacob Alexander Damkjær & Date\\[6ex]
\makebox[2.8in]{\hrulefill} & \makebox[1.8in]{\hrulefill}\\
Anders Ellinge & Date\\[6ex]
 \makebox[2.8in]{\hrulefill} & \makebox[1.8in]{\hrulefill}\\
Sune Straarup Jensen & Date\\[6ex]
 \makebox[2.8in]{\hrulefill} & \makebox[1.8in]{\hrulefill}\\
Jacob Gjervig Strømvig & Date\\[6ex]
 \makebox[2.8in]{\hrulefill} & \makebox[1.8in]{\hrulefill}\\
Magnus Værbak & Date\\[6ex]
\end{tabular}
~
\end{center}
%\vfill
\begin{center}
\today
\end{center}
\newpage
\section*{Chapter 1: Team members and contact persons}
\paragraph{§ 1.}
The following persons constitute team \emph{Dynamic ! Business} and commit to the rules and guidances described in this contract:
\begin{itemize}
\item Anders Ellinge
\item Jacob Alexander Damkjær
\item Jacob Gjervig Strømvig
\item Magnus Værbak
\item Sune Straarup Jensen
\item William Bergmannn Børresen
\end{itemize}

\paragraph{§ 2.}
The following persons act as supervisors and/or consultants for team \\ \emph{Dynamic ! Business}:
\begin{itemize}
\item Ashok Kumar Singh (CTO)
\item Poul Erik Vesterløkke (CEO)
\item Nicolai C. Christensen (SDU Vikings representative)
\end{itemize}

\section*{Chapter 2: General team rules}
\paragraph{§ 3.}
Team meetings are scheduled to be conducted at Thursday afternoons from 12:00 to 15:00. Additional meetings can be scheduled at Monday afternoons from 12:00.
\subparagraph{Section 2.} 
A summary is made for each team meeting. Magnus Værbak is responsible for this task.
\subparagraph{Section 3.}
A notice of meeting which includes an agenda is sent out prior to each team meeting. Magnus Værbak is responsible for this task. 

\paragraph{§ 4.}
Problems and frustrations, personal as well as professional, that affect a team member’s workability or the overall performance of the team have to be discussed as soon as possible, so that solutions can be found, or at least so that the remaining team members are aware of the situation and can plan accordingly. 
\paragraph{§ 5.}
The team expects that the individual team members make an effort to deliver a performance of 100 \% of what they are able to. No specific grade is expected, because it is hard to predict, but the best grade possible is sought achieved.
\newline
\paragraph{§ 6.}
Penalties are incorporated in the team to prevent procrastinations and ensure that the team progresses smoothly towards fulfilling the goals of the project. 
\subparagraph{Section 2.}
If a team member arrives with a delay of more than 15 minutes after the start of a team meeting without notifying the remaining team members of this delay, the member in concern will be imposed a penalty. The member owes cake, a round of beers or something equivalent.
\subparagraph{Section 3.}
If a team member is absent from a team meeting without notifying the remainder of the team, the penalty will be meted out according to \textit{Section 2.}.
\subparagraph{Section 4.}
In case it happens that a team member does not fulfil an important task which he is expected to take care of or if he fails to keep an important agreement or causes major setbacks to the progression of the project, the member in concern will initially receive a warning from the other team members. If a team member has received numerous warnings, the team supervisors will be alerted regarding the specific situation and the team member in question.
\paragraph{§ 7.}
At the end of each group meeting, the GAIT table is updated. The newest version will be available on Google Drive. 
\subparagraph{Section 2.}
The GAIT tool weight for each activity performed will be set by the team members present at the team meeting prior to the beginning of the activity. When the activity has been completed, the weight will be adjusted at the subsequent team meeting if deemed necessary.
\paragraph{§ 8.}
If a decision is to be made between two or more options, a vote can be proposed if the team cannot if deemed necessary. The 
option that gets the most votes will be chosen.
\section*{Chapter 3: Planning of work and time schedules}
\paragraph{§ 9.}
The main time schedule of the tasks expected to be conducted throughout the EiT course can be found in appendix A. Be aware that appendix version is the initial version of the schedule, as the actual schedule will be updated according to the progress of the team and the identification of new tasks. The newest version will always be accessible on Google Drive.
\paragraph{§ 10.}
At every team meeting, the team members consider and update the time schedule and identify the tasks expected to be carried out after the meeting.

\section*{Chapter 4: The learning environment and \\ knowledge sharing}
\paragraph{§ 11.}
To ensure a thriving learning environment and prevent team members from being left behind in any field of knowledge that is included in the project, each team member explains the progress they have made on their assigned task since last team meeting. This explanation might and ought to include a brief teaching session, especially if the progress in question is very specialized and/or complex.

\section*{Chapter 5: Commercial potential} \label{Com}
\paragraph{§ 12.}
If it happens that the project becomes successful enough to constitute an actual commercial opportunity, the team contract will be revised and extended immediately to include an agreement for the situation in question. This revision needs to be conducted in the presence of the entire team (see \textit{§ 14}).

\section*{Chapter 6: Contract validity \\ and change proposals}
\paragraph{§ 13.}
The contract is valid from the date when each member has applied their signatures to the front page. The validity expires at the date when the last team member finishes their EiT examination.
\subparagraph{Section 2.}
The paragraphs in \textit{chapter 5} does not expire at the date mentioned in \textit{section 1.}, but will expire at the time specifically mentioned in \emph{chapter 5}, if any such is mentioned.
\paragraph{§ 14.} \label{Change}
Any proposal for changes to the team contract has to be accepted by \textbf{\textit{every}} team member at a team meeting before they can be incorporated into the team contract. The proposals must be notified to the entire team no later than at the day before the meeting.
\paragraph{§ 15.}
This contract is designated for team \emph{Dynamic ! Business} and acts as an extension to the EiT Handbook. If any conflicts occur between the team contract and the handbook, the rule in the handbook overrules the one in the contract.

Heste

\end{document}